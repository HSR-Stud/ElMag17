\section{Magnetostatische Analyse (MS, Magnetostatic Analysis)}
% Magnetische Analyse wird auch für die Berechnung der Ersatzinduktivität von elektrischen Leitungen verwendet.
%TODO Magnetostatische Analyse
\subsection{Integralgleichungen}
\begin{tabular}{|p{.30\textwidth} |p{.65\textwidth}|}
	\hline 
	\textbf{Ampèresches Gesetz} \newline
	\tabbild[width=3cm]{images/ampgesetz.png} & 
	\[ \oint\limits_{(C)}\vec{H}\cdot\vec{dl} = \iint\limits_{(A)}\vec{J}\cdot\vec{dA}\] \newline
	 \[ \oint\limits_{(C)}\vec{B}\cdot\vec{dl} = \mu_{0}\iint\limits_{(A)}\vec{J}\cdot\vec{dA}\]\\
	\hline
	\textbf{Durchflutungsgesetz} \newline
	& \[ \oint\limits_{(C)}\vec{H}\cdot\vec{dl} = \sum\limits_{k = 1}^{n} I_k = \theta \] \\
	\hline
	\textbf{Quellenfreiheit} \newline
	\tabbild[width=3cm]{images/quellenfreiheit.png} & Der magnetische Fluss durch eine geschlossene Fläche ist immer Null. Somit sind magnetische Feldlinien immer geschlossen. Es gibt keine magnetische Monopole \newline
	\[ \oiint\limits_{(A)}\vec{B}\cdot\vec{dA} = 0\]\\
	\hline
\end{tabular}

\subsection{Differenzialgleichungen der magnetostatischen Analyse}
\subsection{Randbedingungen}
\subsection{Anwendung}
\clearpage
\pagebreak
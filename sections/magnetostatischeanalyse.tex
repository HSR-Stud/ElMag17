\section{Magnetostatische Analyse (MS, Magnetostatic Analysis)}
Die magnetostatische Analyse basiert auf der gleichmässigen Bewegung des elektrische Stroms. Der elektrische Strom ist über die Zeit konstant. Die magnetostatische Analyse wird für die Berechnung der Ersatzinduktivität von elektrischen Komponenten gebraucht 
%TODO Magnetostatische Analyse
\subsection{Integralgleichungen}
\begin{tabular}{|p{.30\textwidth} |p{.65\textwidth}|}
	\hline 
	\textbf{Ampèresches Gesetz} \newline
	{\centering\tabbild[width=4cm]{images/ampgesetz.png}\par} & Das Ampèresche Gesetz definiert die Verteilung des magnetischen Feldes durch eine geschlossene Kurve ist der gesamte Strom durch die entsprechende Fläche
	\[ \oint\limits_{(C)}\vec{H}\cdot\vec{dl} = \iint\limits_{(A)}\vec{J}\cdot\vec{dA}\] \newline
	 \[ \oint\limits_{(C)}\vec{B}\cdot\vec{dl} = \mu_{0}\iint\limits_{(A)}\vec{J}\cdot\vec{dA}\]\\
	\hline
	{\centering\textbf{Durchflutungsgesetz}\par}
	& \[ \oint\limits_{(C)}\vec{H}\cdot\vec{dl} = \sum\limits_{k = 1}^{n} I_k = \theta \] \\
	\hline
	\textbf{Coulombsches Gesetz} \newline
	{\centering\tabbild[width=4cm]{images/quellenfreiheit.png}\par} & Der magnetische Fluss durch eine geschlossene Fläche ist immer Null. Somit sind die magnetische Feldlinien immer geschlossen. Es gibt keine magnetische Monopole. Das magnetische Feld ist quellenfrei \newline
	\[ \oiint\limits_{(A)}\vec{B}\cdot\vec{dA} = 0\]\\
	\hline
\end{tabular}

\subsection{Differenzialgleichungen der magnetostatischen Analyse}

\subsection{Randbedingungen}
\subsection{Anwendung}
\clearpage
\pagebreak
\section{Elektrostatische Analyse (ES, Electrostatic Analysis)}
Die elektrostatische Analyse arbeitet mit dem elektrostatischen (ruhenden) Feld. In diesem Fall ist die elektrische Ladung stationär verteilt (Ladungsverteilung ändert sich nicht). Mittels dieser Analyse kann das elektrische Feld, die Kapazität und die Energie in den elektrischen Komponenten berechnet werden. Dient zur Berechung von Isolatoren. \enlargethispage{\baselineskip}
\subsection{Integralgleichungen}
\begin{tabular}{|p{.30\textwidth} |p{.65\textwidth}|}
	\hline
	\textbf{Gausssches Gesetz}\newline
	{\centering\tabbild[width=3.5cm]{images/Gauss.png}\par}&
	Der Fluss des Vektors $\vec{D} = \varepsilon\cdot\vec{E}$ durch eine geschlossene orientierte Fläche (A) ist gleich der gesamten elektrischen Ladung Q, die von der Fläche (A) umgeben ist.\newline
	\[\varoiint\limits_{(A)}\vec{D}\cdot\vec{dA} = Q \quad oder \quad \varoiint\limits_{(A)}\vec{E}\cdot\vec{dA} = \dfrac{Q}{\varepsilon}\]
    \[ E = \frac{- \diff \varphi}{\diff r} \]
	\\[-0.7cm]
	\hline
	\textbf{Wirbelfreiheit des elektrostatischen Feldes}\newline
	{\centering\tabbild[width=3cm]{images/Wirbelfreiheit}\par}{\centering\tabbild[width=3.5cm]{images/Wirbelfreiheit1}\par}& Das Kurvenintegral des elektrostatischen Feldes $\vec{E}$ über jede geschlossene orientierte Kurve (C) ist gleich null. Das heisst das Kurvenintegral des elektrische Feldes ist nur von der Position des Anfangs- und Endpunkt abhängig. \newline 
	\[\oint\limits_{(C)}\vec{E}\cdot\vec{dl} = 0\] 
	\[\oint\limits_{(C)}\vec{E}\cdot\vec{dl} = \int\limits_{\substack{P_1\\ (C_1)} }^{P_2}\vec{E}\cdot\vec{dl} - \int\limits_{\substack{ P_1\\(C_2)} }^{P_2}\vec{E}\cdot\vec{dl} = 0\]\\
	\hline
	\textbf{Elektrisches Skalarpotential}\newline
	{\centering\tabbild[width = 4cm]{images/Skalarpotential}\par} & Das elektrische Skalarpotential eines Punktes gegenüber dem Bezugspunkt ($P_B$). \newline
	\[\varphi_{P_1} = \int\limits_{P_1}^{P_N}\vec{E}\cdot\vec{dl}\quad und \quad \varphi_{P_2} = \int\limits_{P_2}^{P_N}\vec{E}\cdot\vec{dl}\] \[U_{P_1P_2} = \varphi_{P_1} - \varphi_{P_2} = \int\limits_{P_1}^{P_2}\vec{E}\cdot\vec{dl} \]\\
	\hline
	\textbf{Elektrische Energie}\newline
	Falls eine Umwandlung von kartesischen Koordinaten in Zylinder- oder Kugelkoordinaten nötig ist müssen die Gesetze der Flächen- und Volumenelemente beachtet werden (Bronstein S.540 \& S.546)
	& \[W_{2D}=\frac{1}{2} \iint\limits_{A} \rho \cdot \varphi \, dA \quad \quad W_{3D}=\frac{1}{2} \iiint\limits_{V} \rho \cdot \varphi \, dV \]
	\[W_{2D} = \frac{1}{2} \iint\limits_{A} D \cdot  E \, dA = \frac{1}{2} \iint\limits_{A}\varepsilon\cdot E^2 \, dA\quad \quad
	W_{3D}=\frac{1}{2} \iiint\limits_{V} D \cdot E \, dV
	\]
	\[W_e = \frac{1}{2}\cdot C\cdot U^2\]\\
	\hline
\end{tabular}
\clearpage
\pagebreak
\subsection{Differenzialgleichungen der elektrostatischen Analyse}
\begin{tabular}{|p{.45\textwidth} |p{.45\textwidth}|}
	\hline
	\textbf{Gradient}\newline
	\[ \vec{E}= - \dfrac{\partial\varphi}{\partial x} \cdot \vec{e_{x}} -  \dfrac{\partial\varphi}{\partial y} \cdot \vec{e_{y}}- \dfrac{\partial\varphi}{\partial z} \cdot \vec{e_{z}} = -\nabla \cdot \varphi = -\gradient \varphi\]&
	\textbf{Divergenz}\newline
	\[ \rho= \dfrac{\partial D_{x}}{\partial x} +  \dfrac{\partial D_{y}}{\partial y} + \dfrac{\partial D_{z}}{\partial z}= \nabla \cdot \vec{D}= \divergenz \vec{D} \]\\
	\hline
	\textbf{Poisson-Gleichung}\newline
	\[ \dfrac{\partial^2\varphi}{\partial x^2} +  \dfrac{\partial^2\varphi}{\partial y^2} + \dfrac{\partial^2\varphi}{\partial z^2} =\Delta \varphi = -\dfrac{\rho}{\varepsilon} \]&
	\textbf{Laplace-Gleichung}  \[ \dfrac{\partial^2\varphi}{\partial x^2} +  \dfrac{\partial^2\varphi}{\partial y^2} + \dfrac{\partial^2\varphi}{\partial z^2} =\Delta \varphi = 0 \]\\
	\hline
\end{tabular}
\subsection{Randbedingungen}
\begin{minipage}{8cm}
	\begin{itemize}
		\item Der geerdete Rand \[\varphi(x,y,z) = 0\]
		\item Der Rand mit bekannten Potential \[ \varphi(x,y,z) = A \]
		\item Der Rand der Symmetrie \[ \dfrac{\partial\varphi(x,y,z)}{\partial r} = 0\]
		\item Der Rand zwischen zwei Materialien \[\varphi_{1}=\varphi_{2}\]
		\[ D_1 = D_2 \Rightarrow \varepsilon_1 \cdot E_{1}=\varepsilon_2 \cdot E_{2} \Rightarrow \varepsilon_1\cdot\dfrac{\partial\varphi_1}{\partial x} = \varepsilon_2\cdot\dfrac{\partial\varphi_2}{\partial x}\]
	\end{itemize}
\end{minipage}
\begin{minipage}{8cm}
	\includegraphics[width=8cm]{images/randbedinung_ES.png}
\end{minipage}
\subsection{Randwertproblem}
\begin{multicols}{2}
\begin{itemize}
	\item $\dfrac{\partial^2\varphi}{\partial x^2} +  \dfrac{\partial^2\varphi}{\partial y^2} + \dfrac{\partial^2\varphi}{\partial z^2}=0 $
	\item $\varphi(x,y,z)=0 \in \Gamma_e$
	\item $\varphi(x,y,z)=0 \in \Gamma_b$
	\item $ \dfrac{\partial\varphi(x,y,z)}{\partial r} = 0 \in \Gamma_s$
\end{itemize}
\end{multicols}
\subsection{Vorgehen}
\begin{enumerate}
	\item Partielle Differentialgleichung 2. Ordnung des Potentials aufstellen (Poisson oder Laplace)
	\item Koordinatentransformation wenn nötig
	\item Vereinfachung der partiellen DGL in eine gewöhnliche DGL 
	\subitem Von welcher Variable hängt das Potential ab
	\item Aufstellen der Randbedingungen
	\subitem Randwerte für Potential und E-Feld
	\item Gewöhnliche DGL 2x integrieren und nach Potential auflösen
	\item Randwerte einsetzen und unbekannte Konstanten bestimmen
\end{enumerate}
\clearpage
\pagebreak

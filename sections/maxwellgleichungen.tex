\section{Maxwell Gleichungen}
\begin{longtable}{|p{.30\textwidth} |p{.30\textwidth}|p{.30\textwidth}| }
	\hline
	\textbf{Differentialform} &\textbf{Differentialform} &\textbf{Integralform}\\
	1.Art & 2.Art &\\
	\hline
	$\divergenz \vec{D}=\nabla \cdot \vec{D}= \rho$ & $\divergenz \vec{E}= \frac{\rho}{\epsilon}$&$\oint\limits_{\partial V}\vec{D}\cdot d\vec{A}=\int\int\int\limits_{V}\rho \cdot dV=Q$\\
	\hline
	$\divergenz \vec{B}=\nabla \cdot \vec{B}=0$&$\divergenz \vec{H}=\nabla \cdot \vec{H}=0 $&$\oint\limits_{\partial V}\vec{B}\cdot d\vec{A}=0$\\
	\hline
	$\rotation \vec{E}=\nabla \times \vec{E}=-\frac{\partial B}{\partial t}$&$\rotation \vec{E}=\nabla \times \vec{E}=-\mu \frac{\partial H}{\partial t}$&$\oint\limits_{\partial A}\vec{E}\cdot ds=-\int\int\limits_{A}\frac{\partial B}{\partial t}\cdot dA$\\
	\hline
	$\rotation\vec{B}=\nabla\times\vec{B}=\mu_0 (J+\frac{\partial D}{\partial t})$&$\rotation \vec{H}=\nabla \times \vec{H}=\sigma\vec{E}+\epsilon \frac{\partial E}{\partial t}$&$\oint\limits_{\partial A} \vec{H}\cdot ds$\\
	\hline
\end{longtable}
\subsection{Erstes Maxwell-Gesetz}
\begin{tabular}{p{.45\textwidth} p{.45\textwidth}}
	\textbf{Differentialform}&\textbf{Integralform}\\
	Das E-Feld ist ein Quellenfeld. Die Ladung ist die Quelle des elektrischen Feldes. & Der elektrische Fuss durch die geschlossene Oberfläche eines Volumen ist direkt proportional zur elektrischen Ladung in seinem inneren. \\
\end{tabular}
\subsection{Zweites Maxwell-Gesetz}
\begin{tabular}{p{.45\textwidth} p{.45\textwidth}}
	\textbf{Differentialform}&\textbf{Integralform}\\
	Das B-Feld ist quellenfrei. Es gibt keine magnetische Monopole (Magnet welcher nur ein Pol hat).& Der magnetische Fluss durch die Oberfläche eines Volumen ist gleich der magnetischen Ladung in seinem inneren, nämlich Null\\
\end{tabular}
\subsection{Drittes Maxwell-Gesetz}
\begin{tabular}{p{.45\textwidth} p{.45\textwidth}}
	\textbf{Differentialform}&\textbf{Integralform}\\
	Jede Änderung des B-Feldes führt zu einem elektrischen Gegenfeld. Die Wirbel des elektrischen Feldes sind von der zeitlichen Änderung der magnetischen Flussdichte abhängig (Induktionsgesetz). & Die elektrische Zirkulation ( Umlaufintegral eines Vektorfeldes über  einen geschlossenen Weg) über eine Kurve einer Fläche ist gleich der negativen Änderung des magnetischen Flusses durch die Fläche.\\
\end{tabular}
\subsection{Viertes Maxwell-Gesetz}
\begin{tabular}{p{.45\textwidth} p{.45\textwidth}}
	\textbf{Differentialform}&\textbf{Integralform}\\
	Die Wirbel des Magnetfeldes hängen von der Stromdichte und von der elektrischen Flussdichte ab. Die zeitliche Änderung der Flussdichte wird als Verschiebungsstromdichte bezeichnet (Durchflutungsgesetz)& Die magnetische Zirkulation über eine Kurve einer Fläche ist gleich der Summe aus dem Leitungsstrom und der zeitlichen Änderung des Flusses durch die Fläche. \\
\end{tabular}
\subsection{Elektromagnetische Wellengleichung}
\begin{tabular}{p{.45\textwidth} p{.45\textwidth}}
	\textbf{Magnetisches Feld}&\textbf{Elektrisches Feld}\\
	$\Delta\vec{H}-\mu\sigma\frac{\partial \vec{H}}{\partial t}-\mu\epsilon\frac{\partial^{2}\vec{H}}{\partial t^{2}}=0$&	$\Delta\vec{E}-\mu\sigma\frac{\partial \vec{E}}{\partial t}-\mu\epsilon\frac{\partial^{2}\vec{E}}{\partial t^{2}}=0$  \\
\end{tabular}

\subsection{Randwertproblem}
$\boxed{\Delta\vec{H}-\mu\sigma\frac{\partial \vec{H}}{\partial t}-\mu\epsilon\frac{\partial^{2}\vec{H}}{\partial t^{2}}=0}$\\
$\boxed{\Delta \vec{H}-\imaginär\omega\mu\sigma\vec{H}+\omega^{2}\mu\epsilon\vec{H}=0}$
\clearpage
\pagebreak
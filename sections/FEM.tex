\section{FEM (Finite-Element-Methode)}
\begin{itemize}
	\item Das FEM ist ein Verfahren für die Feldberechnung in einer Simulation (Software)
	\item Die geometrische 2D Anordnung wird mit einem Gitter aus Dreiecken vernetzt. In jedem Dreieck wird das Feld mathematisch approximiert. 
	\item Die geometrische 3D Anordnung wird mit einem Gitter aus Tetraeder vernetzt. In jedem Dreieck wird das Feld mathematisch approximiert. 
	\item Grosse 3D Modelle brauchen viel RAM und CPU-Zeit, weil viele Elemente vorhanden sind und für jedes die Feldgleichungen gelöst werden müssen
	\item Je kleiner die Elemente sind desto genauer kann das Feld berechnet werden. 
	\item Höhere Genauigkeit führt gleichzeitig zu höherem Speicherbedarf und längerer Rechungszeit
\end{itemize}
\clearpage
\pagebreak
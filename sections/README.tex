\thispagestyle{empty}
\setcounter{page}{0} %Set PageNumber to 0
{\huge README }
\section*{Beschreibung}
Formelsammlung für Angewandter Elektromagnetismus Felder und Wellen auf Grundlage der Vorlesung HS 16 von Prof. Dr.Jasmin Smajic \newline
Bei Korrekturen oder Ergänzungen wendet euch an einen der Mitwirkenden.

\section*{Modulschlussprüfung}
Prüfungsstoff ist der gesamte ElMag-Vorlesungsinhalt des HS2016 einschliesslich aller UE.\newline
Als Hilfsmittel für die Modulschlussprüfung sind die Vorlesungen,\newline
UE-Aufgaben und eigenen Aufzeichnungen sowie der Taschenrechner erlaubt.

\subsection*{Plan und Lerninhalte}
{\scriptsize 
\begin{itemize}
    \item Simulationsbasiertes Design und virtuelles Prototyping 
	\item Dielektrisches Design (die elektrostatischen Gleichungen und Randbedingungen)
	\item Elektromechanisches Design (die magnetostatischen Gleichungen und Randbedingungen,  magnetische Kräfte und ihre Kopplung mit der mechanischen Analyse)
	\item Elektrothermisches Design (Eisen- und Kupferverluste und ihre Kopplung mit der thermischen Analyse)
	\item Elektrodynamische Analyse und dazu gehörende Eigenwertprobleme (Maxwell-Gleichungen, Wellenausbreitung, Wellenleiter, Antennen, Resonatoren und Lichtleitfasern) 
	\item Finite-Element-Methode für elektromagnetische Simulationen 
	\item Skalar-FEM (Elektrostatik, Magnetostatik, Wirbelstromanalyse usw.) 
	\item Vektor-FEM (3-D-Wirbelstromanalyse, Wellenausbreitung, Eigenwertprobleme)
	Praktische Anwendungen 
	\item Dielektrische Berechnungen von Hochspannungsgeräten
	\item Wirbelstromanalyse von Leistungstransformatoren
	\item Elektromagnetische Analyse elektrischer Maschinen
	\item Eigenwertanalyse von Wellenleitern und/oder optischen Leitfasern
\item Elektromagnetische Analyse von Mikrowellen- und/oder optischen Antennen
	\item Elektromagnetische Verträglichkeit (EMV).
\end{itemize}
}
\vfill
\section*{Contributors}
\begin{tabular}{ll}
    Stefan Reinli & stefan.reinli@hsr.ch \\ 
    Michel Gisler& michel.gisler@hsr.ch \\ 
\end{tabular} 

{\scriptsize 
\section*{License}
\textbf{Creative Commons BY-NC-SA 3.0}

Sie dürfen:
\begin{itemize}
    \item Das Werk bzw. den Inhalt vervielfältigen, verbreiten und öffentlich
    zugänglich machen.
    \item Abwandlungen und Bearbeitungen des Werkes bzw. Inhaltes anfertigen.
\end{itemize}
Zu den folgenden Bedingungen:
\begin{itemize}
    \item Namensnennung: Sie müssen den Namen des Autors/Rechteinhabers in der von ihm
    festgelegten Weise nennen.
    \item Keine kommerzielle Nutzung: Dieses Werk bzw. dieser Inhalt darf nicht für
    kommerzielle Zwecke verwendet werden.
    \item  Weitergabe unter gleichen Bedingungen: Wenn Sie das lizenzierte Werk bzw. den
    lizenzierten Inhalt bearbeiten oder in anderer Weise erkennbar als Grundlage
    für eigenes Schaffen verwenden, dürfen Sie die daraufhin neu entstandenen
    Werke bzw. Inhalte nur unter Verwendung von Lizenzbedingungen weitergeben,
    die mit denen dieses Lizenzvertrages identisch oder vergleichbar sind.
\end{itemize}
Weitere Details: http://creativecommons.org/licenses/by-nc-sa/3.0/ch/
}
%If we meet some day, 
%and you think this stuff is worth it, you can buy me a beer in return.
\clearpage
\pagenumbering{arabic}% Arabic page numbers (and reset to 1)
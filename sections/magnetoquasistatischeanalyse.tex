\section{Magnetoquasistatische Analyse (MQS, Magnetoquasistatic Analysis)}
%TODO Magnetoquasistatische Analyse
\subsection{Integralgleichungen}
\begin{tabular}{|p{.30\textwidth} |p{.65\textwidth}|}
	\hline 
	\textbf{Ampèresches Gesetz} \newline
	{\centering\tabbild[width=4cm]{images/ampgesetz.png}\par} & Das Ampèresche Gesetz definiert die Verteilung des magnetischen Feldes durch eine geschlossene Kurve ist der gesamte Strom durch die entsprechende Fläche
	\[ \oint\limits_{(C)}\vec{H}\cdot\vec{dl} = \iint\limits_{(A)}\vec{J}\cdot\vec{dA}\] \newline
	\[ \oint\limits_{(C)}\vec{B}\cdot\vec{dl} = \mu_{0}\iint\limits_{(A)}\vec{J}\cdot\vec{dA}\]\\
	\hline
	\textbf{Coulombsches Gesetz} \newline
	{\centering\tabbild[width=4cm]{images/quellenfreiheit.png}\par} & Der magnetische Fluss durch eine geschlossene Fläche ist immer Null. Somit sind die magnetische Feldlinien immer geschlossen. Es gibt keine magnetische Monopole. Das magnetische Feld ist Quellenfrei \newline
	\[ \oiint\limits_{(A)}\vec{B}\cdot\vec{dA} = 0\]\\
	\hline
	\textbf{Faradaysches Induktionsgesetz}
	{\centering\tabbild[width=4cm]{images/faradaygesetz.png}\par} & Der zeitabhängige magnetische Fluss induziert elektrische Spannung in der vom Fluss durchflossenen Spule.\newline
	\[u_{i}=-\frac{\partial \Phi_{m}}{\partial t}\] \newline
	\[\Phi_{m}=\oiint\limits_{(S)} \vec{B}\cdot \vec{dS}\quad und \quad u_{i}=\oint \limits_{(C)}\vec{E}\cdot \vec{dl}\]\newline
	\[\oint \limits_{(C)}\vec{E}\cdot \vec{dl}= - \frac{\partial}{\partial t} \oiint\limits_{(S)} \vec{B}\cdot \vec{dS}\]\\
	\hline
\end{tabular}
\subsection{Differenzialgleichungen der magnetoquasostatischen Analyse}
\subsection{Randbedingungen}
\subsection{Anwendung}
\clearpage
\pagebreak